\chapter{命令行参数}
\section{定义命令行参数}

对于参数,一般使用:
\begin{lstlisting}
	-Xxx[=OPTIONS]
	如果-Xxx后接=,则把=后所有字符视为长参数选项	
\end{lstlisting}

例如:有-Define[=Key=Value]参数:
\begin{lstlisting}
	合法的参数形式:
	-Define=Key=Value
	
	非法的参数形式:
	--Define:Key=Value
	-DefineKey=Value
	Define=Key=Value
	...
\end{lstlisting}
参数命名一般使用大驼峰命名法,即每个单词首字母大写。



\section{定义动词参数}
\subsection{动词参数}
对于一些动词参数,如:
\begin{lstlisting}
	Git Add
	Apt Update
\end{lstlisting}
动词命名使用大驼峰命名法。而且一般情况下只允许存在一个动词参数。动词参数要求早于任何其他参数传入。


\chapter{参数约定}

\section{一些常用参数规定}

\subsection{必选参数}

对于所编写的软件,强制要求拥有以下参数。
\label{_PARAM_HELP_RULE_}
\begin{lstlisting}
	-Help 		打印软件的帮助到标准输出
	-Version 	打印软件的版本号
\end{lstlisting}
如果用户输入了错误的选项,则打印帮助信息(输出应该同`-Help`参数)。
一般情况下,有以上两个选项之一的,就不再进行正常的软件逻辑,忽略其余选项和已经解析过的选项,在打印后即正常退出软件。


\subsection{常见参数释义}
\begin{lstlisting}
	-Debug				启动Debug模式
	-Output[=VALUE]		指定输出
	-Quiet				启动静默模式
	
	如果包含以下参数,那么在执行相应操作后正常退出程序。忽略已解析和未解析的其他参数。
	-Info 				打印软件详细信息,一般包括(版本号,许可证,和其他必要信息)
\end{lstlisting}

