\chapter{命令行参数}
\section{定义命令行参数}





\subsection{短参数}

对于短参数,一般使用:
\begin{lstlisting}
	-X[Options]
	-X后即接参数选项(可选)
	中间无分隔符
	
	短参数不可以合并。如:-D -E 不可写作-DE
\end{lstlisting}

例如:有D E F三个短参数。
-D 短参数指定选项OPT。
\begin{lstlisting}
	合法的参数形式:
	-DOPT -E -F
	
	非法参数形式:
	-DEF
	-D=DOPT -E -F
	-D=DOPT -EF
	DEF
	...
\end{lstlisting}





\subsection{长参数}

对于长参数,一般使用:
\begin{lstlisting}
	--XXX[=OPTIONS]
	如果--XXX后接=,则把=后所有字符视为长参数选项	
\end{lstlisting}

例如:有--define[=KEY=VALUE]参数:
\begin{lstlisting}
	合法的参数形式:
	--define=Key=Value
	
	非法的参数形式:
	--define:Key=Value
	--defineKey=Value
	-define=Key=Value
	define=Key=Value
	...
\end{lstlisting}
短参数和长参数不必对应。如:
-D[Key=Value]
和
--Define[=Key=Value]
可以有不同的意义。


\section{定义动词参数}
\subsection{动词参数}
对于一些动词参数,如:
\begin{lstlisting}
	git add
	apt update
\end{lstlisting}
则动词命名使用大驼峰命名法。而且一般情况下只允许存在一个动词参数。


\chapter{参数约定}

\section{一些常用参数规定}

\subsection{必选参数}

对于所编写的软件,强制要求拥有以下参数,并且下面的参数不受本标准的命名规则等约束(可以在保留的情况下加入符合本条款的对应选项,即-H和--Help):
\begin{lstlisting}
	-h			打印软件的帮助到标准输出。
	--help 		同上
\end{lstlisting}
并且:如果用户输入了错误的选项,则打印帮助信息。帮助信息的内容要引导用户使用-h 或 --help参数。
一般情况下,有以上两个选项之一的,就不再进行正常的软件逻辑,忽略其余选项和已经解析过的选项。在打印帮助后即退出软件。

\subsection{参数命名}
短参数大写。长参数使用大驼峰命名法。如
\begin{lstlisting}
	-D
	-O
	-E
	
	--OutputFile
	--DebugMode
	--DefineVariable
\end{lstlisting}



\subsection{常见参数释义}
\begin{lstlisting}
	--Debug				启动Debug模式
	--Output[=VALUE]	指定输出
	--Quiet				启动静默模式
\end{lstlisting}

